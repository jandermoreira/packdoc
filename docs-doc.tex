%! Author = Jander Moreira
%! Date = 2024

\documentclass[11pt, outputdir = ./out]{article}
\usepackage[T1]{fontenc}

\usepackage{docs}

\title{The \PackageName{docs} package\footnote{This document is refers to the package version \DocsVersion, dated \DocsDate. Compilation date: \today.}}
\author{Jander Moreira -- \texttt{moreira.jander@gmail.com}}
\date{October 27, 2024}

\begin{document}
\maketitle

\tableofcontents


\section{Introduction}
\index{intro}


\section{Package usage}
The package \PackageName{docs}.

\DocsCreate{Package}[color = blue, prefix = \textbackslash]
\begin{Package*}{usepackage}{\OArg{package options}\FArg{docs}}{}
    No \Argument{package options} for now.
\end{Package*}


\section{References and index support}

An \textit{element} in the scope of this document refers to an item that can be highlighted and referenced, such as macros, options and environments, for example.

\DocsCreate{Option}[
color = green!75!black,
index entry = Options,
index remark = {~(option)},
arguments prefix = \texttt{ = },
]
\DocsCreate{Macro}[
prefix = \textbackslash,
color = blue,
index entry = Macros,
index remark = {~(macro)},
]
To instance an element, the \MacroRef{DocsCreate} must be used.

\begin{Macrodef}{DocsCreate}{\MArg{element name}\OArg{element options}}{}
    This macro creates a new element named \Argument{element name} and several other macros to use it.

    \begin{example}{}
        \DocsCreate{MyOption}[color = red]
        This is a declaration \MyOption{scale} is used to scale the box.

        Click to go to \MyOptionRef{scale}.
    \end{example}
\end{Macrodef}

When an element is created, several other macros are created for different needs.

\begin{center}
    \begin{tabular}{lp{4.8cm}l}
        \textbf{Macro} & \textbf{Description} & \textbf{Example} \\
        \hline
        \Macro{\Argument{element name}} & Formats to element style. & \latexinline{\MyOption{a4paper}} \\
        \Macro{\Argument{element name}Def} & Formats to element style, sets a label and index the element. & \latexinline{\MyOptionDef{left}} \\
        \Macro{\Argument{element name}Ref} & Formats to element style and hyperlinks to the definition. & \latexinline{\MyOptionRef{no align}} \\
        \Macro{\Argument{element name}Ind} & Formats to element style and index the element.  & \latexinline{\MyOptionInd{showframe}} \\
        \Macro{\Argument{element name}Index} & Index the element without any typeset. & \latexinline{\MyOptionIndex{element}} \\
        \Macro{\Argument{element name}RefInd} & Formats to element style, hyperlinks to the definition and add an entry to the index. & \latexinline{\MyOptionRefInd{a4paper}} \\
    \end{tabular}
\end{center}

Altogether, two environments are also created.

\DocsCreate{Environment}[
    index entry = Environments,
    index remark = (environment),
    color = orange!70!black,
]

\begin{center}
    \begin{tabular}{lp{7cm}l}
        \textbf{Environment} & \textbf{Description} & \textbf{Example} \\
        \hline
        \Environment{\Argument{element name}def} & Defines a item and creates a label and an index entry. & \latexinline{\begin{MyMacrodef}} \\
        \Environment{\Argument{element name}env} & Defines an environment and creates a label and an index entry. & \latexinline{\begin{MyMacrodef}} \\
        \Environment{\Argument{element name}*} & Defines an element; no labels nor indexing. & \latexinline{\begin{MyMacro*}} \\
    \end{tabular}
\end{center}

\bigskip
These macros and environments are covered by two examples

\subsection*{LinkColors: an element for colors in links}

Suppose you are writing a package such as \PackageName{hyperref} that support some colored features. So you can create an element named \latexinline{LinkColors} to document options to set some colors. This can be done with \Macro{DocsCreate}.

\begin{latexcode}
    \DocsCreate{LinkColors}[color = red]
\end{latexcode}

This command creates several other macros, as described below.

\begin{Macro*}{LinkColors}{\OArg{options}\MArg{text}}{}
    The \Argument{text} is simply expanded to the specified format for the element.
\end{Macro*}

\begin{Macro*}{LinkColorsDef}{\OArg{options}\MArg{text}}{}
    The \Argument{text} expands to the specified format, a label is created and \Argument{text} is inserted in the index.
\end{Macro*}

\begin{Macro*}{LinkColorsRef}{\OArg{options}\MArg{text}}{}
    The \Argument{text} expands to the specified format and a link to where \Argument{text} was defined is inserted.
\end{Macro*}

\begin{Macro*}{LinkColorsInd}{\OArg{options}\MArg{text}}{}
    The \Argument{text} expands to the specified format and \Argument{text} is added to the index. No labels are created.
\end{Macro*}

\begin{Macro*}{LinkColorsIndex}{\OArg{options}\MArg{text}}{}
    \Macro{LinkColorsIndex} inserts an entry for \Argument{text} in the index without producing any output.
\end{Macro*}

\begin{Macro*}{LinkColorsRefInd}{\OArg{options}\MArg{text}}{}
    Extra index entry
\end{Macro*}

\begin{example}{}
    \DocsCreate{LinkColors}[color = red, arguments prefix = \texttt{~=~}]
    The colors in links use the follwing colors:
    \begin{itemize}
        \item \LinkColorsDef{mail} for \texttt{mailto:} links;
        \item \LinkColorsDef{url} for \texttt{http:} and \texttt{https:} links;
        \item \LinkColorsDef{pipe} for \texttt{pipe:} links;
        \item \LinkColorsDef{link} for other links.
    \end{itemize}

    All these colors have default values, such as dark blue for \LinkColorsRef{url}. There is option for citation, such as \LinkColors{citecolor} in the \PackageName{hyperref} package.

    The \LinkColorRef{mail} option is defined below.

    \begin{LinkColorsdef}{mail}{\Argument{color}}{}
        This option sets the color for \texttt{mailto:} links to \Argument{color}.
    \end{LinkColorsdef}
\end{example}



Some options to customize the element appearance are provided.


\begin{Optiondef}{color}{\Argument{color}}{Initially: \latexinline{.!75}}
\end{Optiondef}

\begin{Optiondef}{font}{\Argument{font}}{Initially: \Macro{ttfamily}}
\end{Optiondef}

\begin{Optiondef}{index entry}{\Argument{text}}{Initially: \Argument{element name}}
\end{Optiondef}

\begin{Optiondef}{index remark}{\Argument{text}}{Initially: \Argument{element name}}
\end{Optiondef}

\begin{Optiondef}{prefix}{\Argument{text}}{Initially: \Empty}
\end{Optiondef}

\begin{Optiondef}{arguments prefix}{\Argument{text}}{Initially: \Empty}
\end{Optiondef}

\subsection{Creating elements}


\section{Code examples}

Minted \texttt{outputdir}.


\section{Indexing elements}


\section{Version control}

\printindex

\end{document}