%! Author = Jander Moreira
%! Date = 2024

\documentclass[11pt, outputdir = ./out]{article}
\usepackage[T1]{fontenc}

\usepackage{docs}

\title{The \PackageName{docs} package\footnote{This document is refers to the package version \DocsVersion, dated \DocsDate. Compilation date: \today.}}
\author{Jander Moreira -- \texttt{moreira.jander@gmail.com}}
\date{October 27, 2024}

\begin{document}
\maketitle

\tableofcontents

\DocsNewVersion{0.1}{2024/12/04}

\DocsPrintChanges

\section{Introduction}

\DocsAddChange {0.1}{
    description = {Initial version.},
    no box,
    no listing,
}%
This package is\ldots


\section {Documentation}

\subsection{Package usage and options}
To use the \PackageName{docs} package just load it.

\begin{latexcode}
    \usepackage{docs}
\end{latexcode}


\subsection{Basic commands}

A series of macros are provided to typeset things consistently.

\begin{Macrodef}{PackageName}{\OArg{options}\MArg{name}}{}
    This macro is intended to typeset \Argument{name} as the name of \LaTeX\ packages and classes, such as \PackageName{amsmath} and \PackageName{article}.

    The optional \Argument{options} can be used to locally change the style used for package names by changing \OptionRef{package style}.
\end{Macrodef}

\begin{example}{}
    Useful packages are \PackageName{graphicx} and \PackageName{colorx}, for example. Also \PackageName[package style = \scshape\color{blue}]{babel} and \PackageName{inputenc} are important.
\end{example}

\begin{Macrodef}{Argument}{\OArg{options}\MArg{name}}{}
    The \Macro{Argument} macro is used to typeset arguments.

    With the option \OptionRef{argument color} the color of the argument can be changed. For now, font, size and shape for arguments are hardcoded and cannot be modified.
\end{Macrodef}

\begin{example}{}
    The \PackageName{article} class can use \Argument{options} (see \Cref{sec:options-for-elements}). The paper size is one of these \Argument[argument color = blue]{options}.
\end{example}

\begin{Macrodef}{MArg}{\OArg{options}\MArg{name}}{}
    ``MArg'' stands for \textit{mandatory argument} and is an \MacroRef{Argument} between braces. The same \Argument{options} for \MacroRef{Argument} also apply.
\end{Macrodef}

\begin{Macrodef}{OArg}{\OArg{options}\MArg{name}}{}
    ``OArg'' stands for \textit{optional argument} and is an \MacroRef{Argument} between brackets. The same \Argument{options} for \MacroRef{Argument} also apply.
\end{Macrodef}

\begin{example}{}
    Mandatory argument: \MArg{arg}.\par
    Optional argument: \OArg{arg}.
\end{example}

Additionaly, arguments for arguments between angular brackets (\texttt{<>}) and plain text (\latexinline{{newcounter}}) are also available.

\begin{Macrodef}{AArg}{\OArg{options}\MArg{name}}{}
    ``AArg'' stands for \textit{optional argument between angular brackets}. The same \Argument{options} for \MacroRef{Argument} also apply.
\end{Macrodef}

\begin{example}{}
    Class \PackageName{beamer} can use overlays in slides. For example, \AArg[argument color = red]{range} can be used in a itemized list and \Argument{range} can be set to \texttt{2} (only on slide 2) or \texttt{2-5} (from slide 2 to 5), for example.
\end{example}

\begin{Macrodef}{PArg}{\MArg{name}}{}
    ``PArg'' stands for \textit{mandatory plain text argument} and is an \MacroRef{Argument} between brackets without any special format.
\end{Macrodef}

\begin{example}{}
    Plain argument: \PArg{article}
\end{example}


\subsection{Elements}

An \textit{element} in the scope of this document refers to an item that can be highlighted and referenced, such as macros, options and environments, for example.

To instance an element, the \MacroRef{DocsNewElement} macro must be used.

\DocsNewElement{EnumItemOption}{
    color = red!75!black,
    arguments prefix = \texttt{~=~},
    index heading = \texttt{EnumItemOption} example,
    index remark = (\texttt{EnumItemOption} example),
    no single index,
    no group index,
}
\begin{Macrodef}{DocsNewElement}{\MArg{element name}\MArg{element options}}{}
    This macro creates a new element named \Argument{element name} and several other macros to use it. The \Argument{element options} are a key/value list of options to change how the item will look like.

    \begin{latexcode}
        \DocsNewElement{EnumItemOption}{color = red!75!black}
    \end{latexcode}

    \begin{example}{}
        I like to use the \PackageName{enumitem} package. It makes easier to fine tune the lists appearance, such as using \EnumItemOption{itemsep} or \EnumItemOption{parsep} to change the spaces between the items.
    \end{example}

    When an element is created, several other macros are created for different needs.

    \begin{center}
        \begin{tabular}{lp{4.3cm}l}
            \textbf{Macro}                        & \textbf{Description}                                                                  & \textbf{Ex0.1}                        \\
            \hline
            \Macro{\Argument{element name}}       & Formats to element style.                                                             & \latexinline{\MyElement{a4paper}}       \\
            \Macro{\Argument{element name}Def}    & Formats to element style, sets a label and index the element. & \latexinline{\MyElementDef{left}} \\
            \Macro{\Argument{element name}Ref}    & Formats to element style and hyperlinks to the definition. & \latexinline{\MyElementRef{no align}} \\
            \Macro{\Argument{element name}Ind}    & Formats to element style and index the element.  & \latexinline{\MyElementInd{showframe}} \\
            \Macro{\Argument{element name}RefInd} & Formats to element style, hyperlinks to the definition and add an entry to the index. & \latexinline{\MyElementRefInd{a4paper}} \\
            \Macro{\Argument{element name}Index}  & Index the element without any typeset.                                                & \latexinline{\MyElementIndex{element}}  \\
        \end{tabular}
    \end{center}
\end{Macrodef}

\begin{Macro*}{\Argument{element name}}{\OArg{options}\MArg{item}}{}
    A macro named after \Argument{element name} is created to typeset \Argument{item} in a consistent way. The appearance will follow that defined when \Argument{element name} was created with \Macro{DocsNewElement}, but can be overridden with \Argument{options} (see \Cref{sec:options-for-elements}).
\end{Macro*}

\begin{example}{}
    I like to use the \PackageName{enumitem} package. It makes easier to fine tune lists, such as using \EnumItemOption{itemsep} or \EnumItemOption{parsep} to change the spaces between them items.
\end{example}

\begin{Macro*}{\Argument{element name}Def}{\OArg{options}\MArg{item}}{}
    A macro \Macro{\Argument{element name}Def} is used to define an \Argument{item}, so it can be cross-referenced and have index entries. The definition can be referenced by the \Macro{\Argument{element name}Ref} macro.

    The appearance will follow that defined when \Argument{element name} was created with \Macro{DocsNewElement}, but can be overridden with \Argument{options} (see \Cref{sec:options-for-elements}).

    The definition of an \Argument {item} can be stated with an environment called \hyperref[environmentdef]{\Environment{\Argument {element name}def}} instead.
\end{Macro*}

\begin{example}{}
    I wrote some code to extend the \PackageName{enumitem} package. Now \EnumItemOptionDef{float} can be used to insert a list in a float.
    % The name 'float' has an anchor (label) and entries in the index.
\end{example}

\begin{Macro*}{\Argument{element name}Ref}{\OArg{options}\MArg{text}}{}
    The macro \Macro{\Argument{element name}Ref} typesets the \Argument{item} and creates a link to its definition.

    The appearance will follow that defined when \Argument{element name} was created with \Macro{DocsNewElement}, but can be overridden with \Argument{options} (see \Cref{sec:options-for-elements}).
\end{Macro*}

\begin{example}{}
    Remember that the \EnumItemOptionRef{float} cannot be used if the list is already in a float.
    % 'float' is a link to the definition
\end{example}

\begin{Macro*}{\Argument{element name}Ind}{\OArg{options}\MArg{item}}{}
    The \Macro{\Argument{element name}} macro defines \Argument{item} and inserts entries to the index. Sometimes a secondary index entry is desired, so \Macro{\Argument{element name}Ind} does the job. A reference to the definition is not created.

    The appearance will follow that defined when \Argument{element name} was created with \Macro{DocsNewElement}, but can be overridden with \Argument{options} (see \Cref{sec:options-for-elements}).
\end{Macro*}

\begin{example}{}
    Here we describe some other important information about the  \EnumItemOptionInd{float} option.
    % 'float' has now new index entries, but it's not a link
\end{example}

\begin{Macro*}{\Argument{element name}RefInd}{\OArg{options}\MArg{text}}{}
    The \Macro{\Argument{element name}RefInd} performs the job of both \Macro{\Argument{element name}Ind} and \Macro{\Argument{element name}Ind}, so the index is affected and a reference to the definition is created.

    The appearance will follow that defined when \Argument{element name} was created with \Macro{DocsNewElement}, but can be overridden with \Argument{options} (see \Cref{sec:options-for-elements}).
\end{Macro*}

\begin{example}{}
    Here we describe some other important information about the  \EnumItemOptionRefInd{float} option.
    % 'float' has now new index entries and is also a link
\end{example}

An element can be defined, as previously stated, by calling \Macro{\Argument{element name}Def}. This is handy for inline definitions. An alternative way to define an element is to use an environment also created by \MacroRef{DocsNewElement}. This environment is named \Environment{\Argument{element name}def}.

\begin{Environmentenv}{\Argument{element name}def}{\OArg{options}\MArg{item}\MArg{arguments}\MArg{complement}}{\Argument{element description}}
    \label{environmentdef}
    This environment uses \Argument{element name}'s styles to define an instance named \Argument{item}, along with its \Argument{arguments} and a \Argument{complement}. The \Argument{complement} is any additional text.

    The header of the definition will use the following format:

    \fbox{\Argument{item}\Argument{args prefix}\Argument {arguments}\Argument {complement prefix}\Argument {complement}}

    The values for \Argument {args prefix} and \Argument {complement prefix} are set by \OptionRef{arguments prefix} and \OptionRef{complement prefix} options respectively.

    This environment will create an anchor to \Argument {item} and add it to the index.
\end{Environmentenv}


\begin{example}{}
    % args prefix is \texttt{~=~} and complement prefix is \hfill
    \begin{EnumItemOptiondef}{float}{\texttt{true} | \texttt{false}}{Default: \texttt{true}; initially: \texttt{false}}
        By adding \EnumItemOption{float} to a list, it will be inserted in a float environment.
    \end{EnumItemOptiondef}
    % This definition can be linked with \<element>Def and item is indexed
\end{example}

Another environment is available to just typeset an item, without creating an anchor and not adding entries to the index.

\begin{Environmentenv}{\Argument{element name}*}{\OArg{options}\MArg{item}\MArg{arguments}\MArg{complement}}{\Argument{element description}}
    The \Environment{\Argument{element name}env*} environment has the same behavior as \Environment{\Argument{element name}env}, but no anchor is created and no entry is added to the index.
\end{Environmentenv}

\begin{example}{}
    \begin{EnumItemOption*}{float}{\texttt{true} | \texttt{false}}{Default: \texttt{true}; initially: \texttt{false}}
        By adding \EnumItemOption{float} to a list, it will be inserted in a float environment.
    \end{EnumItemOption*}
\end{example}


\subsection{Options for elements}\label{sec:options-for-elements}

Several options can be used to customize each element. These options are typically specified when the element is created with \MacroRef{DocsNewElement}, but can also be modified with \MacroRef{DocsSetElement}. Options not specified at creation assume predefined default values, which can also be changed with \MacroRef{DocSet}.

\begin{Macrodef}{DocsSetElement}{\MArg{element name}\MArg{option list}}{}
    After created with \MacroRef{DocsNewElement}, options can be changed \textit{a posteriori} with \Macro{DocsSetElement}
\end{Macrodef}

\begin{example}{}
    \DocsNewElement{MyItem}{color = magenta, no single index, no group index}
    An example of MyItem is \MyItem{example}.\par
    \DocsSetElement{MyItem}{color = blue!80!black, font = \slshape}
    This is another one: \MyItem{instance}.
\end{example}

\begin{Optiondef}{package style}{\Argument {commands}}{Initially: \Macro{sffamily}}
    Sets how \MacroRef{PackageName} will typeset classes and package names.
\end{Optiondef}

\begin{Optiondef}{argument color}{\Argument {color}}{Inititally: \texttt{orange!50!black}}
    Sets the color to typeset arguments (see \MacroRef{Argument}).
\end{Optiondef}

\begin{Optiondef}{prefix}{\Argument {text}}{Initially empty}
    When an element is typeset, \Argument {text} is added before the item's name. For example, if an element is created for macros, \Option{prefix} can be set to \Macro{textbackslash}.
\end{Optiondef}

\begin{Optiondef}{arguments prefix}{\Argument{text}}{Initially empty}
    This options sets the text to be put between the item name and its arguments. For macros, for example, it must be empty; for options it can be set to \texttt{=}.

    This element is only typeset if the \Argument{arguments} are not empty (meaning anything with width equal to zero).
\end{Optiondef}

\begin{Optiondef}{complement prefix}{\Argument{text}}{Initially: \Macro{hfill}}
    The contents of \Argument {text} will be inserted between the \Argument {arguments} and the \Argument {complement}.

    This element is only typeset if the \Argument{complement} is not empty (meaning anything with width equal to zero).
\end{Optiondef}

\begin{Optiondef}{font}{\Argument{commands}}{Initially: \Macro{ttfamily}}
    These \Argument {commands} are prepended to every \Argument {item}.
\end{Optiondef}

\begin{Optiondef}{color}{\Argument{color}}{Initially: \texttt{.!75}}
    This sets the color to be used with the \Argument {item}.
\end{Optiondef}

\begin{Optiondef}{index heading}{\Argument {text}}{Initially: \Argument {element name}}
    When an item is defined (\Macro{\Argument{element name}Def} macro or \hyperref[environmentdef]{\Environment{\Argument {element name}env}}), index entries will be grouped under a main entry named \Argument {text}.

    Grouped index entries can be disabled with \OptionRef{no group index}.

    This option is element-specific and will not work as a global option.
\end{Optiondef}

\begin{Optiondef}{no group index}{\texttt{true} | \texttt{false}}{Default: \texttt{true}; initially \texttt{true}}
    This option suppresses adding entries as groups to the index. Single entries are not affected.
\end{Optiondef}

\begin{Optiondef}{index remark}{\Argument {text}}{Default: \texttt{\{\DocsTilde(\Argument{element name})\}}}
    Every index entry will be appended with \Argument {text} the item name.

    Single entries can be removed with \OptionRef{no single index}.

    This option is element-specific and will not work as a global option.
\end{Optiondef}

\begin{Optiondef}{no single index}{\texttt{true} | \texttt{false}}{Default: \texttt{true}; initially \texttt{true}}
    This option suppresses adding single entries to the index. Group entries are not affected.
\end{Optiondef}

\subsection{Preset elements}

When \PackageName{docs} is loaded with the \OptionRef{presets} option, some useful elements are automatically created.

\bigskip
\begin{tabular}{ll}
    \textbf{Element name} & \textbf{Description}\\
    \hline
    \texttt{Option} & To use with options (as those passed within brackets). \\
    \texttt{Macro} & For macros, preceding them with a backslash. \\
    \texttt{Environment} & For general environments. \\
\end{tabular}

\bigskip
This document used these presets.

\begin{example}{}
    The preset elements include \Option{option}, \Macro{macro} and \Environment{environment}.

    For example, \OptionRef{presets} is a package option. The \MacroRef{DocsNewElement} macro is used to create new elements and \Environment{tabular} is a well known environment.
\end{example}


\section{Change history support}

\begin{latexcode}
    \DocsNewVersion{Ex0.1}{2022}
    \DocsNewVersion{Ex0.1a}{2023}
    \DocsNewVersion{Ex0.2}{2024}
\end{latexcode}
\DocsNewVersion{Ex0.1}{2022}
\DocsNewVersion{Ex0.1a}{2023}
\DocsNewVersion{Ex0.2}{2024}

\begin{example}{}
    \DocsAddChange{Ex0.1}{
        description = {Initial \textbf{example} version.},
        no page,  % no page number
        no box,  % no box in margin
    }%
    The version \textbf{Ex} is just an example, released as \textbf{Ex0.1}. Some importante fixes lead to \textbf{Ex0.1a} and new features were added in \textbf{Ex0.2}.
\end{example}

\begin{example}{}
    \DocsAddChange{Ex0.1a}{
        description = {\Macro{ExLang} added to support new languages \textbf{example}.},
    }%
    The \Macro{ExLang} macro is used to allow keywords translation.
\end{example}

\begin{example}{}
    \DocsAddChange{Ex0.1a}{
        description = {\Macro{ExSet} fixed to handle spaces \textbf{example}.},
        updated,
    }%
    Use \Macro{ExSet} to set global options.
\end{example}

\begin{example}{}
    \DocsAddChange{Ex0.1a}{
        description = {\Macro{ExOld} is deprecated \textbf{example}.},
        deprecated,
    }%
    Use \Macro{ExOld} will no longer be updated. Don't use it anymore.
\end{example}

\begin{example}{}
    \DocsAddChange{Ex0.2}{
        description = {\Macro{ExReallyOld} is no longer present in the package \textbf{example}.},
        removed,
    }%
    \Macro{ExReallyOld} was removed. This will break older documents.
\end{example}

\begin{example}{}
    \DocsAddChange{Ex0.1a}[\Option{invisible}]{
        description = {New option: \Option{invisible} \textbf{example}.},
        new,  % optional
    }%
    Package options are \Option{framed}, \Option{nolinks} and \Option{invisible}.
\end{example}

\begin{example}{}
    \DocsAddChange{Ex0.1a}[\Option{framed}]{
        updated,
        no listing,  % this won't appear in the change history
    }%
    Package options are \Option{framed}, \Option{nolinks} and \Option{invisible}.

    \DocsAddChange{Ex0.2}{
        description = {The code for handling CSV lists was completely rewritten \textbf{example}.},
        no box,  % no box in the margin, but description goes the change history
    }%
    All elements in this package use comma-separated lists.
\end{example}

\printindex

\end{document}